\documentclass{beamer}
\usetheme{Berlin}
\usecolortheme{beaver}
\begin{document}
	\title{License to Code}
	\author{Brandon Moore}
	\date{}
	\section{Title Page}
	\frame{\titlepage}
	\section{Table of Contents}
	\begin{frame}
		\frametitle{Table of Contents}
		\tableofcontents[]
	\end{frame}
	\section{What are we talking about?}
	\subsection{Why you should use a license}
	\begin{frame}
		\frametitle{Who should use a license?}
		\pause
		\begin{itemize}[<+->]
			\item Do you want others to contribute?
			\item Do you want others to use and build on your work?
			\item Do you like open source?
			\item Do you want to define what can be done with your code?
		\end{itemize}
		\pause
		\center\LARGE\textbf{Use A License!}
	\end{frame}
	\begin{frame}
		\frametitle{What licenses do for you}
		
	\end{frame}
	\subsection{Licenses defined}
	\begin{frame}
		\frametitle{licenses:}
		
	\end{frame}
	\section{License choices aplenty}
	\subsection{What choices are there?}
	\begin{frame}
		\frametitle{Common licenses}
	\end{frame}
	\subsection{How do I choose a license?}
	\begin{frame}
		\frametitle{What licenses are other people using}
	\end{frame}
	\begin{frame}
		\frametitle{choosealicense.com}
		On creation of a new Github repository, you are prompted to choose a license. For more information, Github provides the helpful site \href{https://choosealicense.com}
	\end{frame}
	\subsection{The land of the esoteric}
	\begin{frame}
		\frametitle{esoteric licenses}
		As with all things open source, there are esoteric options. Whether practical and extreme, or impractical and goofy, esoteric licenses exist. Examples include:
		\begin{itemize}[<+->]
			\item wtfpl
			\item beerware
			\item fight club license
			\item 
	\end{frame}
	\section{Outside the realm of software}
	\begin{frame}
		\frametitle{Creative Commons}
	\end{frame}
	%Good example link: https://esotericsoftware.com/files/license.txt
	%http://www.makeuseof.com/tag/10-ridiculous-eula-clauses-agreed/
\end{document}
